%% Flip-Flop SR  

\def \SRFF (#1)#2#3#4#5#6{%
	% 1st param = shift  or offsets (x, y) 
	% 2nd param = title 
	% 3rd param = X dimension of rect
	% 4th param = Y dimension of rect
	% 5th param = fill color of the block
	% 6th param = ID for pin connections

  \begin{scope}[shift={(#1)}]

    \draw [line width = 1.4, fill = #5] 
		(0.0, 0.0) rectangle (#3, #4);	% rect bloc
	\draw ({#3/2.0}, -0.5) node[]{$ (#2) $};

    \draw  [line width=1.4] 
		(0.0, 0.0 + 0.2) --++ (-0.5, 0.0) coordinate (R#6);	% as input
    \draw  [line width=1.4] 
		(0.0, #4 - 0.2) --++ (-0.5, 0.0) coordinate (S#6);	% input wire
    \draw  [line width=1.4]
		(#3, 0.0 + 0.4) --++(0.5, 0.0) coordinate (QB#6);	% output Q
    \draw  [line width=1.4] 
		(#3, #4 - 0.4) --++(0.5, 0.0) coordinate (Q#6);	% Qbar

	\coordinate (R) at (0.0+0.2, 0.0 + 0.2);
	\coordinate (S) at (0.0+0.2, #4 - 0.2);

	\coordinate (Qb) at (#3-0.2, 0.0 + 0.4);
	\coordinate (Q) at (#3-0.2, #4 - 0.4);		

	\draw (R) node[]{$R$}; 
	\draw (S) node[]{$S$}; 
	\draw (Q) node[]{$Q$}; 
	\draw (Qb) node[]{$\mathrm{\overline{Q}}$}; 	
  \end{scope}
}

% Flip-Flop D, triggered on rising edge
\def\DFFrise (#1)#2#3#4#5#6 {%
	% 1st param = shift or offsets (x, y)
	% 2nd param = title
	% 3rd param = X dimension of rect
	% 4th param = Y dimension of rect
	% 5th param = fill color of the block
	% 6th param = ID for pin connections
	
  \begin{scope}[shift={(#1)}]

	\draw [line width = 1.4, fill = #5] 
		%(0,0) rectangle (1.2, 1.4);			% rect bloc
		(0.0, 0.0) rectangle (#3, #4);			% rect bloc
	

    \draw  [line width=1.4] 
		(0.0, 0.0 + 0.2) --++ (-0.5, 0.0) coordinate (Ck#6); %  as input
    \draw  [line width=1.4] 
		(0.0, #4 - 0.2) --++ (-0.5, 0.0) coordinate (D#6); % input wire
    \draw  [line width=1.4] 
		(#3, 0.0 + 0.4) --++(0.5, 0.0) coordinate (QB#6);	% output Q
    \draw  [line width=1.4] 
		(#3, #4 - 0.4) --++(0.5, 0.0) coordinate (Q#6);	% Qbar

	\coordinate (Ck) at (0.0+0.6, 0.0 + 0.2);
	\coordinate (D) at (0.0+0.2, #4 - 0.2);

	\coordinate (Qb) at (#3-0.2, 0.0 + 0.4);
	\coordinate (Q) at (#3 -0.2, #4 - 0.4);
		
	\draw [line width=1.4] 
		( 0.0, 0.0+0.2-0.1)--++(0.2, 0.1)--++(-0.2, 0.1);  % TRIANGLE  

	\draw (D) node[]{$D$}; 
	\draw (Ck) node[]{$Ck$}; 
	\draw (Q) node[]{$Q$}; 
	\draw (Qb) node[]{$\mathrm{\overline{Q}}$}; 	
  \end{scope}
}


% Flip-Flop D, triggered on falling edge
\def\DFFfall (#1)#2#3#4#5#6 {%
	% 1st param = shift or offsets (x, y)
	% 2nd param = title
		% 3rd param = X dimension of rect
	% 4th param = Y dimension of rect
	% 5th param = fill color of the block
	% 6th param = ID for pin connections
	
  \begin{scope}[shift={(#1)}]

    \draw [line width = 1.4, fill = #5] 
		%(0,0) rectangle (1.2, 1.4);			% rect bloc
		(0.0, 0.0) rectangle (#3, #4);			% rect bloc

    \draw  [line width=1.4] 
		(0.0, 0.0 + 0.2) --++ (-0.5, 0.0) coordinate (Ck#6); %  as input
	\draw [line width=1.2, fill = #5] 
		(-0.1, {#4/2}) circle (0.1);
    \draw  [line width=1.4] 
		(0.0, #4 - 0.2) --++ (-0.5, 0.0) coordinate (D#6); % input wire
    \draw  [line width=1.4] 
		(#3, 0.0 + 0.4) --++(0.5, 0.0) coordinate (QB#6);	% output Q
    \draw  [line width=1.4] 
		(#3, #4 - 0.4) --++(0.5, 0.0) coordinate (Q#6);	% Qbar

	\coordinate (Ck) at (0.0+0.6, 0.0 + 0.2);
	\coordinate (D) at (0.0+0.2, #4 - 0.2);

	\coordinate (Qb) at (#3-0.2, 0.0 + 0.4);
	\coordinate (Q) at (#3 -0.2, #4 - 0.4);
		
	\draw [line width=1.4] 
		( 0.0, 0.0+0.2-0.1)--++(0.2, 0.1)--++(-0.2, 0.1);  % TRIANGLE  

	\draw (D) node[]{$D$}; 
	\draw (Ck) node[]{$Ck$}; 
	\draw (Q) node[]{$Q$}; 
	\draw (Qb) node[]{$\mathrm{\overline{Q}}$}; 	
  \end{scope}
}



% definition de la bascule JK (sans Preset et Clear) 
% Flip-Flop JK without Preset and Clear 
\def\JKFF (#1)#2#3#4#5#6 {%
	% 1st param = shift or offsets (x, y) 
	% 2nd param = title
	% 3rd param = X dimension of rect
	% 4th param = Y dimension of rect
	% 5th param = fill color 
	% 6th param = ID for pin connections
	
  \begin{scope}[shift={(#1)}]
	% rect block 
    \draw [line width=1.4, fill=#5] (0,0) 
		rectangle (#3, #4);
	% EDGE triangle
	\draw [line width=1.2]
		(0.0, {#4/2.0 + 0.15}) --++ (0.2, -0.15) --++(-0.2, -0.15);
	\draw ({#3/2.0}, -0.5) node[]{$ (#2) $};  % text of the block

    \draw [line width=1.5] 
    (0.0, 0.0 + 0.2) --++ (-0.5, 0.0) coordinate (K#6);		% input wire
    \draw [line width=1.5] 
    (0.0, {#4/2.0}) --++({-1*#3}, 0.0) coordinate (CLK#6);  % clock 

    \draw [line width=1.5] 
		(0.0, {#4 - 0.2}) --++ (-0.5, 0.0) coordinate (J#6);	% input 

    \draw [line width=1.5] 
		(#3, 0.0 + 0.4) --++(0.5, 0.0) coordinate (QB#2); % output Q
    \draw [line width=1.5] 
		(#3, #4 - 0.4) --++(0.5, 0.0) coordinate (Q#6);	% Qbar


	\coordinate (K) at (0.0+0.2, 0.0 + 0.2);
	\coordinate (J) at (0.0+0.2, {(#4) - 0.2});
	\coordinate (QB) at (#3-0.2, 0.0 + 0.4);
	\coordinate (Q) at (#3 -0.2, #4 - 0.4);		% Q

	\draw (K) node[]{\small $K$}; 
	\draw (J) node[]{\small $J$}; 
	\draw (Q) node[]{\small$Q$}; 
	\draw (QB) node[]{\small $\overline{Q}$}; 	

  \end{scope}
}


\def \JKFFrise (#1)#2#3#4#5#6 {%
	% 1st param = shift or offset(x, y)
	% 2nd param = title
	% 3rd param = X dimension of rect
	% 4th param = Y dimension of rect
	% 5th param = fill color 
	% 6th param = ID for pin connections

  \begin{scope}[shift={(#1)}]
	% rect bloc
    \draw [line width = 1.4, fill = #5]
		(0.0, 0.0) rectangle (#3, #4);			
	% EDGE triangle
	\draw [line width=1.2]
		(0.0, {#4/2.0 + 0.15}) --++ (0.2, -0.15) --++(-0.2, -0.15);   
   %\draw [line width=1.2, fill = #5] 
		%(-0.1, {#4/2}) circle (0.1);  % inversion symbol
	\draw ({#3/2.0}, -0.5) node[]{$ (#2) $}; 	% text of block

    \draw [line width=1.5] 
		(0.0, 0.0 + 0.2) --++ (-0.5, 0.0) coordinate (K#6); % input wire
    \draw [line width=1.5] 
		(-0.2, {#4/2.0}) --++({-1*#3+0.2}, 0.0) coordinate (CLK#6); % clock

    \draw [line width=1.5] 
		(0.0, #4 - 0.2) --++ (-0.5, 0.0) coordinate (J#6);	% input 

    \draw [line width=1.5] 
		(#3, 0.0 + 0.4) --++(0.5, 0.0) coordinate (QB#6);	% output Q
    \draw [line width=1.5] 
		(#3, #4 - 0.4) --++(0.5, 0.0) coordinate (Q#6); % Qbar


	\coordinate (K) at (0.0+0.2, 0.0 + 0.2);
	\coordinate (J) at (0.0+0.2, {#4 - 0.2});
	\coordinate (QB) at (#3-0.2, 0.0 + 0.4);
	\coordinate (Q) at (#3 -0.2, #4 - 0.4);		% Q

	\draw (K) node[]{\small $K$}; 
	\draw (J) node[]{\small $J$}; 
	\draw (Q) node[]{\small $Q$}; 
	\draw (QB) node[]{\small $\overline{Q}$}; 	

  \end{scope}
}



\def \JKFFfall (#1)#2#3#4#5#6 {%
	% 1st param = shift or offset (x, y)
	% 2nd param = title
	% 3rd param = X dimension of rect
	% 4th param = Y dimension of rect
	% 5th param = fill color 
	% 6th param = ID for pin connections

  \begin{scope}[shift={(#1)}]
	% rect bloc
    \draw [line width = 1.4, fill = #5]
		(0.0 , 0.0) rectangle (#3, #4);			
	% EDGE triangle
	\draw [line width=1.2]
		(0.0, {#4/2.0 + 0.15}) --++ (0.2, -0.15) --++(-0.2, -0.15);   
   \draw [line width=1.2, fill = #5] 
		(-0.1, {#4/2}) circle (0.1);   % inversion symbol
	\draw ({#3/2.0}, -0.5) node[]{$ (#2) $}; 	% text of block

    \draw [line width=1.5] 
		(0.0, 0.0 + 0.2) --++ (-0.5, 0.0) coordinate (K#6); % input wire
    \draw [line width=1.5] 
		(-0.2, {#4/2.0}) --++({-1*#3+0.2}, 0.0) coordinate (CLK#6); % clock

    \draw [line width=1.5] 
		(0.0, #4 - 0.2) --++ (-0.5, 0.0) coordinate (J#6);	% input 

    \draw [line width=1.5] 
		(#3, 0.0 + 0.4) --++(0.5, 0.0) coordinate (QB#6);	% output Q
    \draw [line width=1.5] 
		(#3, #4 - 0.4) --++(0.5, 0.0) coordinate (Q#6); % Qbar


	\coordinate (K) at (0.0+0.2, 0.0 + 0.2);
	\coordinate (J) at (0.0+0.2, #4 - 0.2);
	\coordinate (QB) at (#3-0.2, 0.0 + 0.4);
	\coordinate (Q) at (#3 -0.2, #4 - 0.4);		% Q

	\draw (K) node[]{\small $K$}; 
	\draw (J) node[]{\small $J$}; 
	\draw (Q) node[]{\small $Q$}; 
	\draw (QB) node[]{\small $\overline{Q}$}; 	

  \end{scope}
}


% Latch JK active on 'ON' state
\def\JKLATCHon (#1)#2#3#4#5#6 {%
	% 1st param = shift or offset (x, y)
	% 2nd param = title
	% 3rd param = X dimension of rect
	% 4th param = Y dimension of rect
	% 5th param = fill color
	% 6th param = ID for pin connections

  \begin{scope}[shift={(#1)}]

    \draw [line width = 1.5, fill = #5] 
		(0.0, 0.0) rectangle (#3, #4);			% rect block

	\draw ({#3/2.0}, -0.5) node[]{$ (#2) $}; 			% name of block

    \draw [line width=1.5]
		(0.0, 0.0 + 0.2) --++ (-0.5, 0.0) coordinate (K#6);		% input wire
    \draw [line width=1.5] 
		(0.0, {#4/2.0}) --++(-1*#3, 0.0) coordinate (CLK#6); % clock 

    \draw [line width=1.5] 
		(0.0, #4 - 0.2) --++ (-0.5, 0.0) coordinate (J#6);	% input 

    \draw [line width=1.5] 
		(#3, 0.0 + 0.4) --++(0.5, 0.0) coordinate (QB#6);	% output Q
    \draw [line width=1.5] 
		(#3, #4 - 0.4) --++(0.5, 0.0) coordinate (Q#6);	% Qbar


	\coordinate (K) at (0.0+0.2, 0.0 + 0.2);
	\coordinate (J) at (0.0+0.2, {#4 - 0.2});
	\coordinate (QB) at (#3 - 0.2, 0.0 + 0.4);
	\coordinate (Q) at (#3 - 0.2, #4 - 0.4);		% Q

	\draw (K) node[]{\small $K$}; 
	\draw (J) node[]{\small $J$}; 
	\draw (Q) node[]{\small $Q$}; 
	\draw (QB) node[]{\small $\overline{Q}$}; 	

  \end{scope}
}



% Latch JK active on 'OFF' state
\def \JKLATCHoff (#1)#2#3#4#5#6 {%
	% 1st param = shift or offset (x, y) 
	% 2nd param = title 
	% 3rd param = X dimension of rect
	% 4th param = Y dimension of rect
	% 5th param = fill color

  \begin{scope}[shift={(#1)}]

    \draw [line width=1.5, fill = #5] 
		(0.0, 0.0) rectangle ({#3}, {#4});			% rect block

-	\draw ({#3/2.0}, -0.5) node[]{$ (#2) $}; 		% name of block

    \draw [line width=1.5]
		(0.0, 0.0 + 0.2) --++ (-0.5, 0.0) coordinate (K#6);		% input wire
    \draw [line width=1.5] 
		(0.0, {#4/2.0}) --++({-(#3)}, 0.0) coordinate (CLK#6); % clock 
	\draw [line width=1.2, fill = #5] 
		(-0.1, {#4/2}) circle (0.1);  % inversion symbol 

    \draw [line width=1.5] 
		(0.0, #4 - 0.2) --++ (-0.5, 0.0) coordinate (J#6);	% input 

    \draw [line width=1.5] 
		(#3, 0.0 + 0.4) --++(0.5, 0.0) coordinate (QB#6);	% output Q
    \draw [line width=1.5] 
		(#3, #4 - 0.4) --++(0.5, 0.0) coordinate (Q#6);	% Qbar

	\coordinate (K) at (0.0+0.2, 0.0 + 0.2);
	\coordinate (J) at (0.0+0.2, #4 - 0.2);
	\coordinate (QB) at (#3 - 0.2, 0.0 + 0.4);
	\coordinate (Q) at (#3 - 0.2, #4 - 0.4);		% Q

	\draw (K) node[]{\small $K$}; 
	\draw (J) node[]{\small $J$}; 
	\draw (Q) node[]{\small $Q$}; 
	\draw (QB) node[]{\small $\overline{Q}$}; 	

  \end{scope}
}
