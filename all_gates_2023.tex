%%% modif August 10th 2023 
% AFNOR symbol for NOT and INVERSION are not ROUND but 
% triangles!

%\usepackage{ifthen}  % by D. Carlisle, not needed finally 

\def \GenericORShape #1#2#3#4#5 {
	% 1st param : X width, formerly #3 
	% 2nd param : offset x, formerly \offx 
	% 3rd param : formerly \cxy
	% 4th param : color, formerly #4 
	% 5th param : ch , \ch 
	\path [draw, line width=1.2, fill = #4] ({#5*#1*#3+ #2}, 0.0)
		to [bend right] ({#1*#3 + #2}, {#1*#3/2.0})
		to [bend right] ({#5*#1*#3 + #2}, {#1*#3})
		--++ ({-1*#5*#1*#3}, 0.0)
		.. controls ++({#1/5.0}, {-1*#1/5.0}) and 
		++({#1/5.0}, {#1/5.0}) .. (#2, 0.0)
		-- cycle;
}


\def \GenericXORShape #1#2#3#4#5 {
	% this must have a shorter X width because of eXclusive symbol
	% at the input side: let think it as the total shape
	% including eXclusive symbol is now equal to the shape of OR

	% 1st param : X width, formerly #3 
	% 2nd param : offset x, formerly \offx 
	% 3rd param : formerly \cxy
	% 4th param : color, formerly #4 
	% 5th param : ch , \ch 
	% 6th param : cxy, \cxy % there is no 6th param

	\path [draw, line width=1.2, fill=#4] ({#5*#1*#3 + #2}, 0.0)
		to [bend right] ({#1*#3 + #2}, {#1*#3/2.0})
		to [bend right] ({#5*#1*#3 + #2}, {#1*#3})
		--++ ({-0.5*#5*#1*#3}, 0.0)  % horizontal part 
		.. controls ++({#1/5.0}, {-1*#1/5.0}) and 
		++({#1/5.0}, {#1/5.0})
		.. ({0.5*#5*#1*#3 + #2}, 0.0)
		-- cycle;  % also an horizontal part
		
	\path [draw, very thick, black] ({#2}, #1*#3)
		.. controls ++({#1/5.0}, {-1*#1/5.0}) and 
		++({#1/5.0}, {#1/5.0}) .. ({#2}, 0.0);
}


\def \GenericANDShape #1#2#3 {
	% #1 is \leg
	% #2 is \dY
	% #3 is #6 or FILL color 
	\path [draw, black, line width=1.2, fill=#3]
		(#1, 0.0) -- ({0.6*#2 + #1}, 0.0) 
		.. controls ++(#2*0.5, #2*0.2) and ++ (#2*0.5, -0.2*#2)		
		.. ({0.6*#2+ #1}, #2) -- (#1, #2) -- cycle;
}


\def \GenericOR #1#2#3#4#5#6 {
	% 1st param : xshift	
	% 2nd param : yshift
	% 3rd param : x dimension
	% 4th param : nb of INPUTS 
	% 5th param : ID 
	% 6th param : color 
	
	% 1st param : X width, formerly #3 
	% 2nd param : offset x, formerly \offx 
	% 3rd param : formerly \cxy
	% 4th param : color, formerly #4 
	% 5th param : ch, \ch
	% 6th param : cxy, \cxy 
	% added an 5th param which is the number of INPUTS 
	
	\def \cxy {0.7}   % body ratio width 
	%\def \dY {\cxy*#3}   % height
	\pgfmathsetmacro{\dY}{\cxy*#3}  % HEIGHT 
	%\begin{scope}[xshift = #1 cm, yshift = {#2 cm - \dY/2.0 cm}] 
	\begin{scope}[xshift = #1 cm, yshift = {#2 cm - #3/2*\cxy cm}] 

	\def \ch  {0.3}
	\pgfmathsetmacro {\leg}{0.25*\dY}
	\pgfmathsetmacro {\offx}{0.12*\dY}
	% all is now shifted \offx to the right!
	% a) draw the output line first 
	\draw [line width=1.2, black] ({\dY+\offx}, {\dY/2}) %!!!bdr
		--++(\leg, 0.0) coordinate (OUTOR#5); 
	% b) draw inputs 
	\pgfmathsetmacro {\k}{2*#4}   % k = 2*N (nb of inputs) 
	\foreach \itt in {1, 2,...,#4} {
		\pgfmathsetmacro{\it}{2*\itt-1}
		\draw [line width=1.2, black] ({2*\leg}, {\dY-\it*\dY/\k})
		--++({-2*\leg}, 0.0) coordinate(INOR#5\itt);
	}		
	% c) draw the gate shape == now replaced by 'GenericORShape' 
	% \path [draw, line width=1.2, fill = #4] ({\ch*#3*\cxy+\offx}, 0.0)
		% to [bend right] ({#3*\cxy+\offx}, {(#3)*\cxy/2.0})
		% to [bend right] ({\ch*#3*\cxy+\offx}, {#3*\cxy})
		% --++ ({-1*\ch*#3*\cxy}, 0.0)
		% .. controls ++({#3/5.0}, {-1*#3/5.0}) and 
		% ++({#3/5.0}, {#3/5.0}) .. (\offx, 0.0)
		% -- cycle; 
	\GenericORShape{#3}{\offx}{\cxy}{#6}{\ch}
	\end{scope}
}


\def \GenericNOR #1#2#3#4#5#6 {
	% added an 5th param which is the number of INPUTS 
	\GenericOR{#1}{#2}{#3}{#4}{#5}{#6}
	
	\def \cxy {0.70}   % body ratio width 	
	\begin{scope}[xshift = #1 cm, yshift = {#2 cm - #3/2*\cxy cm}] 
	\pgfmathsetmacro {\dY} {#3*\cxy}
	\pgfmathsetmacro {\leg}{0.3*\dY}
	\pgfmathsetmacro {\offx}{0.12*\dY}
	
	% d) the inversion symbol 
	\draw [line width=1.2, fill=#6] 
		({\dY+\offx+0.25*\leg}, {\dY/2}) circle (0.3*\leg); 
	\end{scope}
}


\def \GenericXOR #1#2#3#4#5#6 {
	% added an nth param which is the number of INPUTS 
	\def \cxy {0.7}   % body ratio width 

	\begin{scope}[xshift = #1 cm, yshift = {#2 cm - #3/2*\cxy cm}]
	\def \ch  {0.3}
 
	\pgfmathsetmacro {\leg}{0.25*#3*\cxy}
	\pgfmathsetmacro {\offx}{0.12*#3*\cxy}
	% all is now shifted \offx to the right!
	% a) draw the output line first 
	\draw [line width=1.2, black] ({#3*\cxy+\offx}, {#3*\cxy/2}) %!!!bdr
		--++(\leg, 0.0) coordinate (OUTOR#5); 
	% b) draw inputs 
	\pgfmathsetmacro {\k}{2*#4}   % k = 2*N (nb of inputs) 
	\foreach \itt in {1, 2,...,#4} {
		\pgfmathsetmacro{\it}{2*\itt-1}
		\draw [line width=1.2, black] ({2*\leg}, {#3*\cxy-\it*#3*\cxy/\k})
		--++({-2*\leg}, 0.0) coordinate(INOR#5\itt);
	}		
	% c) draw the gate shape 
	\GenericXORShape{#3}{\offx}{\cxy}{#6}{\ch}
	\end{scope}
}


\def \GenericXNOR #1#2#3#4#5#6 {
	% % added an 5th param which is the number of INPUTS 
	% after factorization 20th /09/21 
	\GenericXOR{#1}{#2}{#3}{#4}{#5}{#6}
		
	\begin{scope}[xshift = #1 cm, yshift = {#2 cm - #3/2*\cxy cm}]
	\def \cxy {0.7}   % body ratio width 
	\pgfmathsetmacro {\dY} {#3*\cxy} 
	\pgfmathsetmacro {\leg}{0.3*\dY}
	\pgfmathsetmacro {\offx}{0.12*\dY}	
	% d) the inversion symbol 
	\draw [line width=1.2, fill=#6] 
		({\dY + \offx +0.25*\leg}, {\dY/2}) circle (0.3*\leg); 
	\end{scope}
}


% FOR AND and NAND gates 
\def \GenericAND #1#2#3#4#5#6 {
	% 1st param : xshift	
	% 2nd param : yshift
	% 3rd param : x dimension
	% 4th param : nb of INPUTS 
	% 5th param : ID 
	% 6th param : color 

	\def \cxy {0.6}   % body ratio width
	\pgfmathsetmacro {\dY}{\cxy*#3}  % HEIGHT 
	%\begin{scope}[xshift = #1 cm, yshift = {#2 cm - \dY/2.0 cm}] 
	\begin{scope}[xshift = #1 cm, yshift = {#2 cm - #3/2.0*\cxy cm}]
	%\begin{scope}[xshift = #1 cm, yshift = {#2 cm - #3*\cxy/2.0 cm}] %doesn't work
	\def \ch  {0.3}

	\pgfmathsetmacro {\leg}{0.2*#3}
	% a) output leg 
	\draw [line width=1.2, black]
		({0.75*#3}, {\dY/2.0}) --++(0.25*#3, 0.0) coordinate (OUTAND#5);
	
	% b) draw inputs 
	\pgfmathsetmacro {\k}{2*#4}   % k = 2*N (nb of inputs) 
	\foreach \itt in {1, 2,...,#4} {
		\pgfmathsetmacro{\it}{2*\itt-1}
		\draw [line width=1.2, black] 
			(\leg, {\dY - \it*\dY/\k}) --++ (-1*\leg, 0.0)	
			coordinate(INAND#5\itt);
	}

	% this shape is 0.7 * WIDTH , extra legs are 0.15*2=0.3 off 
	\GenericANDShape{\leg}{\dY}{#6}
	% now replaced by 'GenericANDShape
	% \path [draw, black, line width=1.2, fill=#6]
		% (\leg, 0.0) -- ({0.6*\dY + \leg}, 0.0) 
		% .. controls ++(\dY*0.5, \dY*0.2) and ++ (\dY*0.5, -0.2*\dY)		
		% .. ({0.6*\dY+ \leg}, \dY) -- (\leg, \dY) -- cycle;
		% % to [bend right] ({0.80*#3}, {\dY/2.0})
		% % to [bend right] ({0.6*\dY+ \offx}, \dY) -- (\offx, \dY)
		% % -- cycle; 
	
	\end{scope} 
}


%FOR AND and NAND gates 
\def \GenericNAND #1#2#3#4#5#6 {

	% 1st param : xshift	
	% 2nd param : yshift
	% 3rd param : x dimension
	% 4th param : nb of INPUTS 
	% 5th param : ID 
	% 6th param : color 
	\GenericAND{#1}{#2}{#3}{#4}{#5}{#6}

	\def \cxy {0.6}   % body ratio width
	%\pgfmathsetmacro {\dY}{\cxy*#3}  % HEIGHT 
	\begin{scope}[xshift = #1 cm, yshift = {#2 cm - #3/2.0*\cxy cm}]
	\pgfmathsetmacro {\leg}{0.2*#3}
	\draw [line width=1.2, fill=#6] 
		({\dY + \leg +0.25*\leg}, {\dY/2}) circle (0.3*\leg); 
	\end{scope}
}


%%%%%%%%%%%%%%%%%%%%%%%%%%%%%%%%%%%%%%%%%%%%%%%%%%%%%%%%%%%%%%%%%%%d
%%%%%%%%%%%%%%%%%%%%%%%%%%%%%%%%%%%%%%%%%%%%%%%%%%%%%%%%%%%%%%%%%%%d
%%%%%%%%%%%%%%%%%%%%%%%%%%%%%%%%%%%%%%%%%%%%%%%%%%%%%%%%%%%%%%%%%%%d
%%%%%%%%%%%%%%%%%%%%%%%%%%%%%%%%%%%%%%%%%%%%%%%%%%%%%%%%%%%%%%%%%%%
%%%%%%%%%%%%%%%%%%%%%%%%%%%%%%%%%%%%%%%%%%%%%%%%%%%%%%%%%%%%%%%%%%%
%%% GENERIC GATES : ALL SHAPE SHARE THE SAME FORM FACTOR 
%%% GENERIC GATES : ALL SHAPE SHARE THE SAME FORM FACTOR 
%%%%%%%%%%%%%%%%%%%%%%%%%%%%%%%%%%%%%%%%%%%%%%%%%%%%%%%%%%%%%%%%%%%
%%%%%%%%%%%%%%%%%%%%%%%%%%%%%%%%%%%%%%%%%%%%%%%%%%%%%%%%%%%%%%%%%%%
%%%%%%%%%%%%%%%%%%%%%%%%%%%%%%%%%%%%%%%%%%%%%%%%%%%%%%%%%%%%%%%%%%%
%%%%%%%%%%%%%%%%%%%%%%%%%%%%%%%%%%%%%%%%%%%%%%%%%%%%%%%%%%%%%%%%%%%
%%%%%%%%%%%%%%%%%%%%%%%%%%%%%%%%%%%%%%%%%%%%%%%%%%%%%%%%%%%%%%%%%%%

% \def \ExpliciteGateShape #1#2#3#4#5 {  % formerly
\def \SimpleRectBlock #1#2#3#4#5 {
	% simple rect bloc 
	% #1: X dimension 
	% #2: cxy dimension 
	% #3: offx : offset x 
	% #4: BOX Name, as node 
	% #5: fill color
	\path [draw, black, line width = 1.2, fill = #5] 
		(#3, 0.0) rectangle +({#1 - 2*#3}, {#1*#2});
	% label 
	\node [ ] (t) at ({#1/2.0}, {#1*#2/2.0}) {\small #4};
}


\def \ExpliciteGate #1#2#3#4#5#6#7 {
	% 1st and 2nd param : shift 
	% 3rd  : X dimensions 
	% 4th : TYPE : OR/AND/etc. 
	% 5th: nb of inputs 
	% 6th: ID 
	% 7th: color fill
	\def \cxy {0.5}   % body ratio width
	\begin{scope}[xshift = #1 cm, yshift = {#2 cm - #3/2*\cxy cm}]
	\pgfmathsetmacro {\dY}{\cxy*#3}  % HEIGHT 
	\pgfmathsetmacro {\leg}{0.2*#3} 
	%  a) draw output 
	\draw[line width=1.2, black] 
		(#3 - \leg, {\dY/2.0}) --++ (\leg, 0.0)
		coordinate(out#6) ;
	 % b) draw inputs 
	%\ifthenelse{#5 == 1}
	%{
		%\draw [line width=1.2, black] 
			%(\leg, {\dY/2.0}) --++ (-1*\leg, 0.0)	
				%coordinate(OIG#6);
	%}
	%{
	\pgfmathsetmacro {\k}{2*#5}   % k = 2*N (nb of inputs) 
	\foreach \itt in {1,...,#5} {
		% the classic {1,2,...,#5} is giving one more leg for #5==1
		\pgfmathsetmacro{\it}{2*\itt-1}
		\draw [line width=1.2, black] 
			(\leg, {\dY - \it*\dY/\k}) --++ (-1*\leg, 0.0)	
				coordinate(in#6\itt);
			%	coordinate(IOR#6\itt);
	}
	%}
	
	%\ExpliciteGateShape{#3}{\cxy}{\leg}{#4}{#7}  % formerly 
	\SimpleRectBlock{#3}{\cxy}{\leg}{#4}{#7}  %string don't pass
	\end{scope}
}

%% august 10th 2023
\def \ExpliciteNonGateShape #1#2#3#4#5#6#7 {
	% this is AFNOR implementation of NOT Gate, that is: "porte NON"
	% 1st and 2nd param : shift 
	% 3rd  : X dimensions 
	% 4th : TYPE : OR/AND/etc. 
	% 5th: nb of inputs 
	% 6th: ID 
	% 7th: color fill
	%\ExpliciteOneInputGate{#1}{#2}{#3}{#4}{#5}{#6}{#7}
	\ExpliciteGate{#1}{#2}{#3}{#4}{#5}{#6}{#7}

	\begin{scope}[xshift = #1 cm, yshift = {#2 cm - #3/2*\cxy cm}]
	\pgfmathsetmacro {\dY}{\cxy*#3}  % HEIGHT 
	\pgfmathsetmacro {\leg}{0.2*#3} 
	\def \cxy {0.5}   % body ratio width
	% d) The inversion symbol, instead of a round we put a 
	% oblique bar 
	\draw [line width=1.2, fill=#7] 
		({#3 - \leg + 0.5*\leg}, {\dY/2.0}) --++ 
			({-0.5*\leg}, {0.5*\leg});
	%\draw [line width=1.2, fill=#7] 			
	%	({#3 - \leg + 0.3*\leg}, {\dY/2}) circle (0.3*\leg);	
	\end{scope}
}


% after a bug found for ONE INPUT gate
% this was created for 
\def \ExpliciteOneInputGate #1#2#3#4#5#6 {
	% 1st and 2nd param : shift 
	% 3rd  : X dimensions 
	% 4th : TYPE : OR/AND/etc. 
	%%% 5th: nb of inputs 
	% 5th: ID 
	% 6th: color fill
	\def \cxy {0.5}   % body ratio width
	\begin{scope}[xshift = #1 cm, yshift = {#2 cm - #3/2*\cxy cm}]
	\pgfmathsetmacro {\dY}{\cxy*#3}  % HEIGHT 
	\pgfmathsetmacro {\leg}{0.2*#3} 
	%  a) draw output 
	\draw[line width=1.2, black] 
		(#3 - \leg, {\dY/2.0}) --++ (\leg, 0.0)
		coordinate(out#5) ;
	 % b) draw inputs 
	%\pgfmathsetmacro {\k}{2*#5}   % k = 2*N (nb of inputs) 
	% \foreach \itt in {1, 2,...,#5} {
		% \pgfmathsetmacro{\it}{2*\itt-1}
		\draw [line width=1.2, black] 
			(\leg, {\dY/2.0}) --++ (-1*\leg, 0.0)	
				coordinate(OIG#5);
	% }
	\ExpliciteGateShape{#3}{\cxy}{\leg}{#4}{#6}  %string don't pass
	\end{scope}
}



\def \ExpliciteNotGateShape #1#2#3#4#5#6#7 {
	% 1st and 2nd param : shift 
	% 3rd  : X dimensions 
	% 4th : TYPE : OR/AND/etc. 
	% 5th: nb of inputs 
	% 6th: ID 
	% 7th: color fill
	\ExpliciteGate{#1}{#2}{#3}{#4}{#5}{#6}{#7}

	\begin{scope}[xshift = #1 cm, yshift = {#2 cm - #3/2*\cxy cm}]
	\pgfmathsetmacro {\dY}{\cxy*#3}  % HEIGHT 
	\pgfmathsetmacro {\leg}{0.2*#3} 
	\def \cxy {0.5}   % body ratio width
	% d) The inversion symbol
	\draw [line width=1.2, fill=#7] 
		({#3 - \leg + 0.3*\leg}, {\dY/2}) circle (0.3*\leg);	
	\end{scope}
}


%%%%%%%%%%%%%%%%%%%%%%%
%%%%%%%%%% EN %%%%%%%%
%%%%%%%%%%%%%%%%%%%%%%%
%%%%%%%%%%%%%%%%%%%%%%%
\def \GateShape #1#2#3#4#5 {
	% for YES, AMpli, NOT gate , leg is 0.3*{X dim}
	% so shape is 0.4*{X dim} = dX, and dY < dX, let it be 0.3*{X dim} 
	% #1 = x dimension
	% #2 = cxy  
	% #3 = x shift  
	% #4 = box name, as node text 
	% #5 = fill color 
	\path [draw, line width=1.2, black, fill=#5]
		(#3, 0.0) --++({0.4*#1}, {0.5*#2*#1}) 
		--++ ({-0.4*#1}, {0.5*#2*#1}) -- cycle; 
	\node [] (t) at ({#1/2.0}, #1*#2/2.0) {\small #4};
}


\def \YesGate #1#2#3#4#5#6 {
	% #1 = xshift 
	% #2 = yshift 
	% #3 = x dimension 
	% #4 = ID 
	% #5 = text mark
	% #6 = fill 
	\def \cxy {0.4}   % body ratio width
	\begin{scope}[xshift = #1 cm, yshift = {#2 cm - #3/2*\cxy cm}]
	\pgfmathsetmacro {\dY}{\cxy*#3}  % HEIGHT 
	\pgfmathsetmacro {\leg}{0.3*#3} 
	%  a) draw output 
	\draw[line width=1.2, black] 
		(#3 - \leg, {\dY/2.0}) --++ (\leg, 0.0)
		coordinate (out#4);
	
	% b) input [ONE] 
	\draw[line width=1.2, black] 
		(\leg, {\dY/2.0}) --++ (-1*\leg, 0.0)
		coordinate (in#4);
	\GateShape{#3}{\cxy}{\leg}{#5}{#6} 
	\end{scope}
}


\def \NotGate #1#2#3#4#5#6 {
	% ANGLOPHONE
	% #1 = xshift 
	% #2 = yshift 
	% #3 = x dimension 
	% #4 = ID 
	% #5 = text mark
	% #6 = fill 
	\YesGate{#1}{#2}{#3}{#4}{#5}{#6} 
	\begin{scope}[xshift = #1 cm, yshift = {#2 cm - #3/2*\cxy cm}]
	\def \cxy {0.4}
	\pgfmathsetmacro {\dY}{\cxy*#3}  % HEIGHT 
	\pgfmathsetmacro {\leg}{0.3*#3} 
	\def \cxy {0.5}   % body ratio width
	% d) The inversion symbol
	\draw [line width=1.2, fill=#6] 
		({#3 - \leg + 0.2*\leg}, {\dY/2}) circle (0.2*\leg);	
	\end{scope}
}


%%  August 8th 2023, after noting an error for ONE input
%% explicit gate, finished this today!
\def \ExpliciteYesGate #1#2#3#4#5#6 {
	% FOR Exceptional YES (or NOT) gate! 
	% #1 is X shift, #2 is Y shift
	% #3 is X width
	% #4 is ID 
	% inspired by \ExpliciteGate, takes the place of #6
	% #5 is label (string label), takes the place of #4
	% #6 is fill color, takes the place of #7
	\def \cxy {0.5}   % body ratio width
	\begin{scope}[xshift = #1 cm, yshift = {#2 cm - #3/2*\cxy cm}]
	\pgfmathsetmacro {\dY}{\cxy*#3}  % HEIGHT 
	\pgfmathsetmacro {\leg}{0.2*#3} 
	% a) draw the output 
	\draw[line width=1.2, black] 
		(#3 - \leg, {\dY/2.0}) --++ (\leg, 0.0) coordinate(out#4) ;
	% b) draw THE ONLY one input !!!
	\draw[line width = 1.2, black]
		(0.0, {\dY/2}) --++ (\leg, 0.0) coordinate (in#4);
	%\ExpliciteGateShape{#3}{\cxy}{\leg}{#5}{#6}  %it orks but not 
	\SimpleRectBlock{#3}{\cxy}{\leg}{#5}{#6}  %it orks but not 
	\end{scope}
}


\def \ExpliciteNotGate #1#2#3#4#5#6 {
	% FOR Exceptional Explicite NOT gate! 
	% #1 is X shift, #2 is Y shift
	% #3 is X width
	% #4 is ID 
	% inspired by \ExpliciteGate, takes the place of #6
	% #5 is label (string label), takes the place of #4
	% #6 is fill color, takes the place of #7
	\def \cxy {0.5}   % body ratio width
	\begin{scope}[xshift = #1 cm, yshift = {#2 cm - #3/2*\cxy cm}]
	\pgfmathsetmacro {\dY}{\cxy*#3}  % HEIGHT 
	\pgfmathsetmacro {\leg}{0.2*#3} 
	% a) draw the output 
	\draw[line width=1.2, black] 
		(#3 - \leg, {\dY/2.0}) --++ (\leg, 0.0) coordinate(out#4) ;
	% b) draw THE ONLY one input !!!
	\draw[line width = 1.2, black]
		(0.0, {\dY/2}) --++ (\leg, 0.0) coordinate (in#4);
	\SimpleRectBlock{#3}{\cxy}{\leg}{#5}{#6}  %string don't pass
	% d) The inversion symbol
	\draw [line width=1.2, fill=#6] 
		({#3 - \leg + 0.3*\leg}, {\dY/2}) circle (0.3*\leg);
	\end{scope}
}

